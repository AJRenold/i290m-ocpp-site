
%%%%%%%%%%%%%%%%%%%%%%% file typeinst.tex %%%%%%%%%%%%%%%%%%%%%%%%%
%
% This is the LaTeX source for the instructions to authors using
% the LaTeX document class 'llncs.cls' for contributions to
% the Lecture Notes in Computer Sciences series.
% http://www.springer.com/lncs       Springer Heidelberg 2006/05/04
%
% It may be used as a template for your own input - copy it
% to a new file with a new name and use it as the basis
% for your article.
%
% NB: the document class 'llncs' has its own and detailed documentation, see
% ftp://ftp.springer.de/data/pubftp/pub/tex/latex/llncs/latex2e/llncsdoc.pdf
%
%%%%%%%%%%%%%%%%%%%%%%%%%%%%%%%%%%%%%%%%%%%%%%%%%%%%%%%%%%%%%%%%%%%


\documentclass[runningheads,a4paper]{llncs}

\usepackage{amssymb}
\setcounter{tocdepth}{3}
\usepackage{graphicx}

\usepackage{url}
%\urldef{\mailsa}\path|{alfred.hofmann, ursula.barth, ingrid.haas, frank.holzwarth,|
%\urldef{\mailsb}\path|anna.kramer, leonie.kunz, christine.reiss, nicole.sator,|
%\urldef{\mailsc}\path|erika.siebert-cole, peter.strasser, lncs}@springer.com|    
\newcommand{\keywords}[1]{\par\addvspace\baselineskip
\noindent\keywordname\enspace\ignorespaces#1}

\begin{document}

\mainmatter  % start of an individual contribution

% first the title is needed
\title{Open Source Community \\ Joining \& Management}

% a short form should be given in case it is too long for the running head
\titlerunning{Open Collaboration and Peer-Production}
%\thanks{Please note that the LNCS Editorial assumes that all authors have used
%the western naming convention, with given names preceding surnames. This determines
%the structure of the names in the running heads and the author index.}


% the name(s) of the author(s) follow(s) next
%
% NB: Chinese authors should write their first names(s) in front of
% their surnames. This ensures that the names appear correctly in
% the running heads and the author index.
%
\author{i290m Open Collaboration and Peer-Production Class \inst{1}}
%\and Spencer Wheatley \inst{2} \and Didier Sornette \inst{2}}
%
%%
%%\authorrunning{Maillart et al.}
%% (feature abused for this document to repeat the title also on left hand pages)
%
%% the affiliations are given next; don't give your e-mail address
%% unless you accept that it will be published
\institute{School of Information,\\
UC Berkeley,\\ Berkeley, United States}
%%\mailsa\\
%%\mailsb\\
%%\mailsc\\
%%\url{http://www.springer.com/lncs}
%\and Chair of Entrepreneurial Risks,\\ ETH Zurich, Switzerland}
%
%%
% NB: a more complex sample for affiliations and the mapping to the
% corresponding authors can be found in the file "llncs.dem"
% (search for the string "\mainmatter" where a contribution starts).
% "llncs.dem" accompanies the document class "llncs.cls".
%

%\toctitle{Lecture Notes in Computer Science}
\tocauthor{Authors' Instructions}
\maketitle


\begin{abstract}

Here is the abstract

\keywords{open collaboration, peer-production, community joining, governance, education}
\end{abstract}
\section{Introduction}

Open collaboration and peer production systems comprise a significant
part of the Internet's infrastructure and content.
Practicioners within these systems recognize the importance of
\emph{community}\footnote{Fogel, K., Producing Open Source Software, \url{http://producingoss.com/en/technical-infrastructure.html}} in developing and sustaining these resources.
Paradigmatically, an open source software project is created and maintained by an on-line community.
The community engages in productive dialog using communication tools and contributes to
a common pool \cite{ostrom1990}, in a process that is sometimes called ``private-collective innovation'' \cite{vonhippel2003oss}.
It has been suggested that the core features of such collaborative
communities extend to other communities of practice, such as those
surrounding wikis, open data, and citizen science, as well \todo{by whom?}.

Our work is a contribution to the understanding of  open collaborative
communities and the underlying peer-production labor organization \cite{benkler2002} \todo{don't now what this reference tries to say}.
A critical aspect is the joining and management of these communities. There exists previous efforts in these domains, which have studied joining into these communities and impacts of the joining \cite{vonKrogh2003,Baldwin2006} \todo{more refs would be nice}.
We also focus on the joining process and on how
it integrates in the broader issue of community management.
We contribute by exploring this issue from a novice aspect and by examining non-coding communities, such as citizen science. The former aspect is important, as novel recent developers have different kind of problems than those experienced by the developers \cite{Begel2008}. The latter represent an emerging form of open collaboration, and worth of studies as such.

Our context for this study is a semester long course at UC Berkeley's School of Information, entitled ``Open Collaboration and Peer Production'' (i290m). The course itself applied many open source principles, for example
the students and instructors wrote this report in collaboration.
The students have spent the semester participating in an open
collaborative community of their choosing and reporting
their observations.
We have collaboratively developed a survey about our experiences joining
these projects, and the projects' organization and demographics. The total number of responders is 24, who are also taking part in the report writing.
By administering this survey to ourselves, we have collected data
that samples across a wide range of communities and researcher experiences.

\mysubsubsection{Discovering a Group to Join}
One of the central objectives of this report is to investigate factors that bring projects and potential contributors together. Prior to successfully joining and contributing to a project, newcomers must become aware of projects that are active and open to (or actively seeking) help, with available opportunities that resonate with the potential contributor's motivation to be part of an open source project. In the survey discussed here, we posed and analyzed questions relating to project outreach, contributor incentives, and the quality of the joining experience in order to gain novel perspective on the relationships between these factors for newcomers to a range of project types.\todo{I would just remove whole section}

%This report has also been \emph{written} collaboratively.
%The 24 respondents to our survey are all also authors of this paper.
%We have attempted to organize ourselves according
%to the best practices of open collaborative communities as discussed in literature.

In Section 2 we elaborate on the context of how we have written this report, as an exploration of this novel method is one of our main research contributions.
Section 3 provides some background literature which we draw from in
our survey design and analysis.
Section 4 explains our research methods, including quantitative
reporting and survey design.
Section 5 presents the results of our empirical work.
We discuss implications of our work in Section 6, and conclude. 



\section{Context}

Here is the context

\section{Background}
A decade ago, open source software has appeared as an original self-organized ``Bazaar" way of producing software by opposition to commonly known ``Cathedral" top-down management \cite{raymond1999}. Like many surprising new phenomena it has also attracted research in management \cite{vonkrogh2006pro}, economics \cite{lerner2002}, entrepreneurial legal studies \cite{benkler2002},  quantitative sociology \cite{} and even in complex systems physics \cite{}. Open source software development has been understood as a private-collective 
innovation model in which developers gain both from their own (private) contributions as well as the (collective) work done by others \cite{vonhippel2003oss}. Despite notorious arising successes (e.g. Linux, Apache, Mozilla) and early recognition that fast growing Internet threats could be tackled better with full-disclosure (i.e. open source code) approaches to find and fix security holes in software (by opposition to non-disclosure  proprietary source code) \cite{}, it remained unclear how open source could prevail on proprietary business models.  The (apparent) benevolent commitment of large communities of developers was the most striking questions \cite{}. It is probably still the most fundamental question \cite{benkler2011leviathan}. But since the open source model has developed far beyond software programming as open collaboration for natural language knowledge production (e.g. Wikipedia), for industrial designs (3d printing, cars) \cite{raasch2009,pearce2012}, we at least know that open collaboration works well and probably far beyond most initial expectations. 

As companies get attracted by  ``bazaar" open collaboration and peer-production models for innovation \cite{} and even production \cite{hamel2011first}, research tends to shift interest towards creating and maintaining user communities \cite{vonHippel2001} in closer collaboration with business  \cite{bonaccorsi2004ais} or more recently even completely integrated solutions within company boundaries \cite{}. For instance, one important requirement for getting a job at Facebook is precisely to have some open source software development experience, not only because the quality  of the code produced by the candidate can be checked {\it ex-ante}, but also because it helps build a sense of community.



Socialization in an open source software community: A socio-technical analysis
N Ducheneaut - Computer Supported Cooperative Work (CSCW), 2005 - Springer




%\cite{vonHippel2005Democratizing,}
%The open source software phenomenon: Characteristics that promote research \cite{ vonkrogh2007}
  	

Ten years later we are in this future depicted ten years ago. Yet open collaboration has indeed emerged as an alternative approach to (knowledge) production, one of the most critical issues for each open source project concerns reaching a critical mass of users who have a chance to keep on the increasing needs of a growing project. 

Two fundamental issues are having people be involved at first ({\it joining}), and keep them involved through comprehensive {\it governance} to keep.

{\bf [more on joining here]}

motivations

Comparing motivations of individual programmers and firms to take part in the open source movement: From community to business
A Bonaccorsi, C Rossi - Knowledge, Technology \& Policy, 2006 - Springer


{\bf [more on governance here]} 

 \cite{O'Mahony2007}

(negative) impact of governance on joining \cite{halfacker2013}

Because of the fundamental interplay between joining and governance we explore both joining and governance. But governance mainly from the viewpoint of joiners, as the main goal of the i290m class was intended to help acquire an experience in joining open source projects.

We don't address the production issues which have been raised previously in the literature.


\cite{tomlinson2012}


\section{Methods}

\subsubsection{Survey Design Process}
To complement our qualitative understanding of the {\it joining process} we decided to design and run a survey. Each of us proposed a survey question and posted it on a Google Form \footnote{\url{https://docs.google.com/forms/d/1KnkSkM3f_QBRQeYfUXpyXspSqavjcClKS2DoWAz1fyE/viewform}}. We then categorized questions by assigning tags : communication, governance, contributions, social networks, joining (to be completed). One or more tags could be assigned to one question and has allowed to draw a bipartite network of relationships between questions (c.f. Figure, to be completed).

Once the first survey design was completed each of us took it. We debriefed the result in class and found several flaws that we described in a separate paragraph. However, the advantage of designing a survey in a collective way helped ensure that most questions would be relevant to most of us.

This is a starting point of course and in the future {\bf this survey could be iteratively improved}



\subsection{Qualitative Reporting}

The qualitative reporting is an integrative part of the class in the form of blog post assignments ($1000\pm500$ words per post). Assignments have covered the following topics by chronological order to allow for progressive mutual sharing of experience on the class website \cite{classweb2013}:

\begin{enumerate}
  \item {\bf First Contact with the Community} \\ 
This topic covers the motivations to choose a specific project as well as the process through which joining has occurred \cite{lakhani2005htu,robert2006,vonKrogh2012,}. Since the class organization (c.f. Section \ref{classmotivations}) set no constraints on which project could be chosen a broad variety of experiences where expected. Some reporting was expected on whether joining had been an informal process or a more formal established ``joining" script \cite{vonkrogh2003} and if this experience generalizes to all community joiners.\\

  \item {\bf Historical, Cultural and Demography Backgrounds} \\
Ideological, cultural and historical traits play a fundamental role in {\it sorting} communities. Evidence of this sorting has been brought regarding open source license types \cite{belenzon2009}. The evolution of an open source project and the condition under which it can be joined are path dependent. Deciding to join a project requires a good understanding of the cultural traits and social norms of the community. In this part, it was asked to report on the most relevant historical, cultural and demographic traits of joined projects and communities. \\


 \item{\bf Communication Infrastructure}\\
The success of open collaboration is deeply rooted in the capacity of community to organize with the help of online information systems (c.f. \cite{benkler2002} for theoretical argument), which in turn determine the way communities interact and keep records of the open collaboration innovation steps. \\

  \item {\bf Community Participation} \\
Blog post about community participation. Incorporate links to your project participation and engage the readings. Do they generalize to your experience? Or not? How? \\

  \item {\bf Governance \& Decision Process}\\
  One of toughest issue with large and rather horizontal communities is the decision process. The relevant question here is : How does a community make tough decisions? Open collaboration communities have shown some level creativity in that regard, with a large variety of governance models ranging from consensus ranging to benevolent dictatorship. The governance approach might be related to the project goals or simply be the result of path dependent history. Depending on the community, the governance structure can be informal or on the contrary clearly documented.

\end{enumerate}


\noindent Two blog posts (assignments) are due after the expected delivery of this report so their analysis cannot be included here :

\begin{enumerate}[resume]
\item {\bf Funding Sources :} Where does the funding for you community come from? Is there corporate sponsorship? A foundation that backs it? Do users donate? How does this affect the community's cooperative dynamics? Are there competing projects? How would you describe your project's role in the greater technical ecosystem?

\item{\bf Organization \& Project Modularity :} Does your project's community mirror the technical modularity of the project? How does it structure its collaboration--synchronously? Asynchronously? How does it get work done?

\end{enumerate}

\noindent The full assignment descriptions can be found on the class website \cite{classweb2013}.



\subsection{Survey Design}

\subsubsection{Survey Design Process}
To complement our qualitative understanding of the {\it joining process} we decided to design and run a survey. Each of us proposed a survey question and posted it on a Google Form \footnote{\url{https://docs.google.com/forms/d/1KnkSkM3f_QBRQeYfUXpyXspSqavjcClKS2DoWAz1fyE/viewform}}. We then categorized questions by assigning tags : communication, governance, contributions, social networks, joining (to be completed). One or more tags could be assigned to one question and has allowed to draw a bipartite network of relationships between questions (c.f. Figure, to be completed).

Once the first survey design was completed each of us took it. We debriefed the result in class and found several flaws that we described in a separate paragraph. However, the advantage of designing a survey in a collective way helped ensure that most questions would be relevant to most of us.

This is a starting point of course and in the future {\bf this survey could be iteratively improved}



\mysubsubsection{Survey Design}


To complement our qualitative understanding, we designed a survey {\it in collaboration} among class members. Overall, the questions reflected the course  readings and the participants' experiences within their respective open source communities. 

First, class participants generated questions in a lab session, with several participants incorporating additional questions remotely later on.  Then, the questions were categorized by concept tags: Communication, Demographics, Joining, Infrastructure, Motivation, Governance, Contribution, Finance, Social Network and Educational Experiences. From there, a effort was made to format the survey's  presentation and flow, informed by the categories, as well as reduce replications inquiries.

Questions used in the survey  are presented in Appendix \ref{app:survey_form}. As the survey was intended for class-use only, the class participants did not pilot it before collecting  responses. The final sample size was 24 responses, nn \% of the class \todo{TODO: FIXME}.

In the review of the question and during the analysis phase, we found several flaws described in a separate paragraph (see section \ref{link_to_limitations}\todo{fixme}). We believe these questions can and should be developed further; the current survey is a starting point and {\bf should be iteratively improved}. Some of the appropriate follow-up questions are included in this work. Furthermore, the collective design process ensured that most of questions were relevant to the majority, as can be seen in the following analysis part.
\section{Results}








\section{Discussion}

Here is the Discussion

\documentclass[a4paper,12pt]{article}
\begin{meta-analysis}

Commentary on the process of writing this report: I’m lost. I can’t see what others are writing and I have no perspective on how this paragraph will fit into the larger whole. Will it just fit? Will it be meaningful? Will it exist in a flow of thought? All of this makes me wonder how far we need to come with collaborative writing and coding tools. There needs to be a better vantage point of what others are doing and what needs to be done. GitHub provides some of this, but it’s not in real time. Google docs is in real time, but it’s uncontrolled chaos if 30 people are writing. What would a system in the middle look like?

\end{meta-analysis}


\section{Conclusion}

Here is the Conclusion


This first survey of a limited number of projects suggests some interesting directions for future study. Experiences of students trying to join both very new projects (only a year or two old) and more established projects seemed to yield different results. 

We would like to see future research directed towards illuminating the differences in joining experiences between participants who take part in projects of different ages and stages. 

One participant in a very young project explained: 

“...it feels much more manageable to eventually gain an overall grasp of both the overall structure and many details for this type of early-stage project than if I were to try to contribute to a large, long-running and already widely used project.”





%\input{intro}



\bibliographystyle{abbrv}
\bibliography{references.bib}








%\section*{Appendix: Springer-Author Discount}
%
%LNCS authors are entitled to a 33.3\% discount off all Springer
%publications. Before placing an order, the author should send an email, 
%giving full details of his or her Springer publication,
%to \url{orders-HD-individuals@springer.com} to obtain a so-called token. This token is a
%number, which must be entered when placing an order via the Internet, in
%order to obtain the discount.
%
%\section{Checklist of Items to be Sent to Volume Editors}
%Here is a checklist of everything the volume editor requires from you:
%
%
%\begin{itemize}
%\settowidth{\leftmargin}{{\Large$\square$}}\advance\leftmargin\labelsep
%\itemsep8pt\relax
%\renewcommand\labelitemi{{\lower1.5pt\hbox{\Large$\square$}}}
%
%\item The final \LaTeX{} source files
%\item A final PDF file
%\item A copyright form, signed by one author on behalf of all of the
%authors of the paper.
%\item A readme giving the name and email address of the
%corresponding author.
%\end{itemize}

\end{document}
