\section{Data and Methods}

We developed a method to gain broad insight on the open source project joining process and how the management is likely to help or impede on boarding. We used both quantitative and qualitative analysis in our work, and applied the ideas of method triangulation, i.e. looking the same phenomenas from different perspectives. As the report has been written in collaboration, we suggest that the process of writing has also supported interaction, especially in qualitative analysis and interaction between qualitative and quantitative analysis.

To support our analysis, a diverse set of data, both qualitative and quantitative was collected. First data set emerged from the \textbf{class assigments}, which asked us to describe our participation to open source projects (c.f. Section \ref{sec:qualitative}). In total this data is 45,000 words of qualitative descriptions. The second, more elaborated effort, to collect data was in a form of a \textbf{survey}. The design process is described below (see \ref{sec:survey}), and it included both quantitative and qualitative questions. The total sample of this survey is $n=24$. Last, as majority of the projects used Github to manage the collaboration, by examining the commit logs and other details via \textbf{Github API} created an additional source of data.

%% Our method is based on {\it qualitative self-reporting} guided by class assignments (c.f. Section \ref{opencollaborationintheclassroom}) on the one hand and on a {\it quantitative survey} designed by and tested in the class on the other hand. The overall process is ``declarative" {\bf [Find more appropriate term here]} both in design and results. For instance, the survey was made of questions that students felt important to ask to their fellow classmates. Thus, the study design is deeply endogenous.

In the following sections we describe in detail the data and process how the data was formed. First, each of the data sources is explored (Sections \ref{sec:qualitative} -- \ref{sec:survey}) followed by a short discussion of methods used (Sections \todo{fixme}).

