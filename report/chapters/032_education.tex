\subsection{Open Collaboration in the Classroom}
\label{opencollaborationintheclassroom}

From preliminary work including personal engagement in an open collaboration project and individual assignments, one main goal of the class is to produce a relevant collective report with no {\it ex-ante} guidelines and with self-organization as the default more. Instructors act as ``benevolent dictators" only when required or if extensive discussion are likely the impede to finish the report on time. We report on the main (self-)organization steps for this report.

\begin{enumerate}
  \item collective report topic brainstorming {\bf self-organization} : recall the date and what has triggered a change in the syllabus
  \item decision to design a survey {\bf organization} : by the instructors
  \item survey design {\bf self-organization} : question design (one by student) + categories
  \item survey answering {\bf organized} : everyone had to take the survey within a precise time window (recall it)
  \item survey analysis {\bf organized} :  each of us was asked to perform an analysis of her own proposed question(s).
  \item report Latex format {\bf organization + self-organization} : choice for Latex by the instructors
  \item learning Latex {\bf self-organization} : no crash course $\rightarrow$  learning by doing
  \item group formation to handle parts of the report {\bf self-organized}
  \item production by groups {\bf self-organized} : drafts on Google Docs
  \item peer-review {\bf self-organized} : group rotation
  \item editing harmonization {\bf organized} : few rules (use ``we", present tense by default)
\end{enumerate}

\mysubsubsection{Other patterns of self-organization :} 

Various communication tools used on a best match way : Etherpad, Google Docs, Google Forms 
Make a list of various ad-hoc documents.


Instructors become increasingly community managers