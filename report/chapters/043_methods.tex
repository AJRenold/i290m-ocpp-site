\subsection{Methods}
\label{sec:methods}

\mysubsubsection{Quantitative data}

A set of standard statistical methods have been used in this work to share some light into our analysis, however as our sample size is extremely small ($n=24$), extra care is needed to confirm our analysis results\todo{rewrite?}.

\begin{description}
\item[Descriptive statistics], such as means or distributions of values, are one of the most often used tools in this work.
\item[Cross-tabulation ($\chi^2$)] is method used to confirm if cross-tabulated data is distributed evenly or unevenly. Due to sample size, we have not used the standart $\chi^2$ test, rather we applied \textit{Fisher's exact test}, which unlike the $\chi^2$ test does not assume large enough pool of data. We however report this test as $\chi^2$ and related values in $\chi^2$ format for compatibility.
\item[Correlation] is used to explore detailed relationship between variables. When necessarily, such as in cases of Likert-scales, rank-correlation was used in the calculations
\item[Linear regression] was used to explore the relation of a set of (dependable) variables to one (in-dependabt) variable. To counter problems of small sample size, we report only adjusted R$^2$, however we have not engaged in detailed residual analysis.
\end{description}

For the more analytic statistics, present a $p$-value, which indicates our trust in the examined phenomena. We have accepted values below the 10 \%-cut point as indicators of some level of trust\footnote{This level is indeed accepted in social sciences.}, and indicate the exact value in detail, or just non-significance (n.s.) of the result. This said, results should be considered indicator and as worth of further studies.

\mysubsubsection{Qualitative data}

Even while there is several forms of qualitative methods and theories how to intereprent that kind of data, our work applies the most often used \textiit{content classification} approach, where content is read and re-read by the researchers and classified by their relevance to the research question. In out work certain quotes from the data are used to provide examples of the conclusions made from the qualitative data.