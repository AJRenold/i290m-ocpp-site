\subsection{Qualitative Reporting}

The qualitative reporting is an integrative part of the class in the form of blog post assignments ($1000\pm500$ words per post). Assignments have covered the following topics by chronological order to allow for progressive mutual sharing of experience on the class website \cite{classweb2013}:

\begin{enumerate}
  \item {\bf First Contact with the Community} \\ 
This topic covers the motivations to choose a specific project as well as the process through which joining has occurred \cite{lakhani2005htu,robert2006,vonKrogh2012,}. Since the class organization (c.f. Section \ref{classmotivations}) set no constraints on which project could be chosen a broad variety of experiences where expected. Some reporting was expected on whether joining had been an informal process or a more formal established ``joining" script \cite{vonkrogh2003} and if this experience generalizes to all community joiners.\\

  \item {\bf Historical, Cultural and Demography Backgrounds} \\
Ideological, cultural and historical traits play a fundamental role in {\it sorting} communities. Evidence of this sorting has been brought regarding open source license types \cite{belenzon2009}. The evolution of an open source project and the condition under which it can be joined are path dependent. Deciding to join a project requires a good understanding of the cultural traits and social norms of the community. In this part, it was asked to report on the most relevant historical, cultural and demographic traits of joined projects and communities. \\


 \item{\bf Communication Infrastructure}\\
The success of open collaboration is deeply rooted in the capacity of community to organize with the help of online information systems (c.f. \cite{benkler2002} for theoretical argument), which in turn determine the way communities interact and keep records of the open collaboration innovation steps. \\

  \item {\bf Community Participation} \\
Blog post about community participation. Incorporate links to your project participation and engage the readings. Do they generalize to your experience? Or not? How? \\

  \item {\bf Governance \& Decision Process}\\
  One of toughest issue with large and rather horizontal communities is the decision process. The relevant question here is : How does a community make tough decisions? Open collaboration communities have shown some level creativity in that regard, with a large variety of governance models ranging from consensus ranging to benevolent dictatorship. The governance approach might be related to the project goals or simply be the result of path dependent history. Depending on the community, the governance structure can be informal or on the contrary clearly documented.

\end{enumerate}


\noindent Two blog posts (assignments) are due after the expected delivery of this report so their analysis cannot be included here :

\begin{enumerate}[resume]
\item {\bf Funding Sources :} Where does the funding for you community come from? Is there corporate sponsorship? A foundation that backs it? Do users donate? How does this affect the community's cooperative dynamics? Are there competing projects? How would you describe your project's role in the greater technical ecosystem?

\item{\bf Organization \& Project Modularity :} Does your project's community mirror the technical modularity of the project? How does it structure its collaboration--synchronously? Asynchronously? How does it get work done?

\end{enumerate}

\noindent The full assignment descriptions can be found on the class website \cite{classweb2013}.


