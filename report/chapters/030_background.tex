\section{Background}

A decade ago, open collaboration has appeared as an original way to produce software \cite{}, then natural language knowledge (e.g. Wikipedia) , industrial designs (3d printing, cars) \cite{pearce2012}, and even military robots \footnote{\url{http://www.darpa.mil/Our_Work/TTO/Programs/DARPA_Robotics_Challenge.aspx}} . And a few successes have attracted theoretical research on so-called peer-production and fueled hope that open collaboration would become common place organization design in the near future \cite{benkler2002}. 

economics \cite{lerner2002}

(developer) communist (Olson 1965)

private-collective model (von Hippel and von Krogh, 2003)


Socialization in an open source software community: A socio-technical analysis
N Ducheneaut - Computer Supported Cooperative Work (CSCW), 2005 - Springer




Ten years later we are in this future depicted ten years ago. Yet open collaboration has indeed emerged as an alternative approach to (knowledge) production, one of the most critical issues for each open source project concerns reaching a critical mass of users who have a chance to keep on the increasing needs of a growing project. 

Two fundamental issues are having people be involved at first ({\it joining}), and keep them involved through comprehensive {\it governance} to keep.

{\bf [more on joining here]}

motivations

Comparing motivations of individual programmers and firms to take part in the open source movement: From community to business
A Bonaccorsi, C Rossi - Knowledge, Technology & Policy, 2006 - Springer


{\bf [more on governance here]} 

 \cite{O'Mahony2007}

(negative) impact of governance on joining \cite{halfacker2013}

Because of the fundamental interplay between joining and governance we explore both joining and governance. But governance mainly from the viewpoint of joiners, as the main goal of the i290m class was intended to help acquire an experience in joining open source projects.

We don't address the production issues which have been raised previously in the literature.


