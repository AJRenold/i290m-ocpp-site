\section{Background}
A decade ago, open source software has appeared as an original self-organized ``Bazaar" way of producing software by opposition to commonly known ``Cathedral" top-down management \cite{raymond1999}. Like many surprising new phenomena it has also attracted research in management \cite{vonkrogh2006pro}, economics \cite{lerner2002}, entrepreneurial legal studies \cite{benkler2002},  quantitative sociology \cite{} and even in complex systems physics \cite{}. Open source software development has been understood as a private-collective 
innovation model in which developers gain both from their own (private) contributions as well as the (collective) work done by others \cite{vonhippel2003oss}. Despite notorious arising successes (e.g. Linux, Apache, Mozilla) and early recognition that fast growing Internet threats could be tackled better with full-disclosure (i.e. open source code) approaches to find and fix security holes in software (by opposition to non-disclosure  proprietary source code) \cite{}, it remained unclear how open source could prevail on proprietary business models.  The (apparent) benevolent commitment of large communities of developers was the most striking questions \cite{}. It is probably still the most fundamental question \cite{benkler2011leviathan}. But since the open source model has developed far beyond software programming as open collaboration for natural language knowledge production (e.g. Wikipedia), for industrial designs (3d printing, cars) \cite{raasch2009,pearce2012}, we at least know that open collaboration works well and probably far beyond most initial expectations. 

As companies get attracted by  ``bazaar" open collaboration models for innovation and even production \cite{}, research tends to shift interest towards creating and maintaining user communities \cite{ vonHippel2001} in closer collaboration with business  \cite{bonaccorsi2004ais} or more recently even completely integrated solutions within company boundaries \cite{}.

Socialization in an open source software community: A socio-technical analysis
N Ducheneaut - Computer Supported Cooperative Work (CSCW), 2005 - Springer




\cite{vonHippel2005Democratizing,}

The open source software phenomenon: Characteristics that promote research \cite{ vonkrogh2007}
  	


Ten years later we are in this future depicted ten years ago. Yet open collaboration has indeed emerged as an alternative approach to (knowledge) production, one of the most critical issues for each open source project concerns reaching a critical mass of users who have a chance to keep on the increasing needs of a growing project. 

Two fundamental issues are having people be involved at first ({\it joining}), and keep them involved through comprehensive {\it governance} to keep.

{\bf [more on joining here]}

motivations

Comparing motivations of individual programmers and firms to take part in the open source movement: From community to business
A Bonaccorsi, C Rossi - Knowledge, Technology \& Policy, 2006 - Springer


{\bf [more on governance here]} 

 \cite{O'Mahony2007}

(negative) impact of governance on joining \cite{halfacker2013}

Because of the fundamental interplay between joining and governance we explore both joining and governance. But governance mainly from the viewpoint of joiners, as the main goal of the i290m class was intended to help acquire an experience in joining open source projects.

We don't address the production issues which have been raised previously in the literature.


\cite{tomlinson2012}

