\section{Syllabus}

\textbf{8/30 No class}

\textbf{9/3 Course overview LAB 0}

Skim: Practical OSS Exploration: How to be Productively Lost, the Open
Source Way
\href{http://quaid.fedorapeople.org/TOS/Practical_Open_Source_Software_Exploration/html/}{link}

\textbf{9/6 Collective Action, Incentives, and Performance Evaluation -
Thomas}

Required :

\begin{itemize}
\itemsep1pt\parskip0pt\parsep0pt
\item
  Benkler, Y. The Penguin and the Leviathan: How Cooperation Triumphs
  over Self-Interest (Crown Business, 2011), 1 edn.
  \href{http://www.youtube.com/watch?v=dPbE3WieoUo}{video}
\item
  Gulley, N. \& Lakhani, K. R. The determinants of individual
  performance and collective value in Private-Collective software
  innovation. Social Science Research Network Working Paper Series
  (2010). \href{http://ssrn.com/abstract=1550352}{article}.
\item
  Pickard, G. et al. Time-Critical social mobilization. Science 334,
  509-512 (2011). \href{http://arxiv.org/abs/1008.3172}{preprint}
\item
  Dan Pink \href{http://www.youtube.com/watch?v=KgGhSOAtAyQ}{video}
\end{itemize}

Optional :

\begin{itemize}
\itemsep1pt\parskip0pt\parsep0pt
\item
  Benkler, Y. Coase's penguin, or, linux and ``the nature of the firm''.
  The Yale Law Journal 112, 369+ (2002).
  \href{http://dx.doi.org/10.2307/1562247}{link}.
\item
  Hess, C. \& Ostrom, E. Ideas, Artifacts, and Facilities: Information
  as a Common-Pool Resource. Law and Contemporary Problems 111-146
  (2003).
\item
  Cooper, S. et al. Predicting protein structures with a multiplayer
  online game. Nature 466, 756-760 (2010).
\item
  Ostrom, E. Governing the Commons: The Evolution of Institutions for
  Collective Action (Political Economy of Institutions and Decisions)
  (Cambridge University Press, 1990)
\end{itemize}

\textbf{9/10 Github and Pull Requests LAB 1}

\href{http://courses.ischool.berkeley.edu/i290m-ocpp/site/article/lab1.html}{LAB
1}

\textbf{9/13 Technical Infrastructure and Community Etiquette - Seb}

Required:

\begin{itemize}
\itemsep1pt\parskip0pt\parsep0pt
\item
  Fogel, K. ``Technical Infrastructure'' Producing Open Source Software
  \href{http://producingoss.com/en/technical-infrastructure.html}{link}
\item
  Raymond, E. ``How to ask questions the smart way''
  \href{http://www.catb.org/esr/faqs/smart-questions.html}{link}
\item
  von Krogh, G., Spaeth, S. \& Lakhani, K. R. Community, joining, and
  specialization in open source software innovation: a case study.
  Research Policy 32, 1217-1241 (2003).
  \href{http://flosshub.org/system/files/rp-vonkroghspaethlakhani.pdf}{link}
\end{itemize}

\textbf{9/17 IRC Field Trip - LAB 2}

\href{http://courses.ischool.berkeley.edu/i290m-ocpp/site/article/lab2.html}{LAB
2}

\emph{ASSIGNMENT 1: Report on your experience contacting and explaining
your intention to contribute to the community. Did you follow a
`script'? Was it a formal or informal process? Did you face any barriers
to participation? Do you think your experience was personal or general?
Please include in your post links to opening conversations with the open
project. Include any relevant mailing list posts, wiki edits, and
forum/issue tracker contributions.}

\textbf{9/20 Social Networks, Human Timing, Triggering Dynamics,
Emotions and Politeness - Thomas}

Required :

\begin{itemize}
\itemsep1pt\parskip0pt\parsep0pt
\item
  Leskovec, J., Huttenlocher, D. \& Kleinberg, J. Predicting positive
  and negative links in online social networks. In Proceedings of the
  19th international conference on World wide web, WWW '10, 641-650
  (ACM, New York, NY, USA, 2010).
  \href{http://arxiv.org/abs/1003.2429}{article}
\item
  Maillart, T., Sornette, D., Frei, S., Duebendorfer, T. \& Saichev, A.
  Quantification of deviations from rationality with heavy tails in
  human dynamics. Physical Review E 83, 056101+ (2011).
  \href{http://arxiv.org/abs/1007.4104}{preprint}
\item
  Crane, R. \& Sornette, D. Robust dynamic classes revealed by measuring
  the response function of a social system. Proceedings of the National
  Academy of Sciences 105, 15649-15653 (2008).
  \href{http://arxiv.org/abs/0803.2189}{preprint}
\end{itemize}

Optional :

\begin{itemize}
\itemsep1pt\parskip0pt\parsep0pt
\item
  Danescu-Niculescu-Mizil, C., Sudhof, M., Jurafsky, D., Leskovec, J. \&
  Potts, C. A computational approach to politeness with application to
  social factors. In Proceedings of ACL (2013).
  \href{http://www.mpi-sws.org/~cristian/Politeness.html}{link}
\item
  Mitrović, M., Paltoglou, G. \& Tadić, B. Networks and emotion-driven
  user communities at popular blogs. The European Physical Journal B -
  Condensed Matter and Complex Systems 77, 597-609 (2010).
  \href{http://link.springer.com/article/10.1140/epjb/e2010-00279-x}{article}
\item
  Granovetter, M. S. The strength of weak ties. American Journal of
  Sociology 78, 1360-1380 (1973).
  \href{http://sociology.stanford.edu/people/mgranovetter/documents/granstrengthweakties.pdf}{Article}
\end{itemize}

\textbf{9/24 Hunting Social Epidemics \& Human Timing (Data exploration)
LAB 3}

\href{http://courses.ischool.berkeley.edu/i290m-ocpp/site/article/lab3.html}{LAB
3}

\textbf{9/27 Intellectual and Institutional History - Seb}

Required:

\begin{itemize}
\itemsep1pt\parskip0pt\parsep0pt
\item
  ``Brief history of open source''
  \href{http://www.netc.org/openoptions/background/history.html}{link}
\item
  Stallman, R. ``Why Software Should Not Have Owners.''
  \href{http://www.gnu.org/philosophy/why-free.html}{link}
\item
  \href{http://www.ischool.berkeley.edu/newsandevents/news/20120216timoreilly}{O'Reilly,
  T.} ``The Architecture of Participation''
  \href{http://oreillynet.com/pub/wlg/3017}{link}.
\item
  (just the introduction) O'Reilly, T. ``Government as a Platform''
  \href{http://chimera.labs.oreilly.com/books/1234000000774/ch01.html}{link}
\item
  skim Newitz, A. ``I've Seen the Worst Memes of My Generation Destroyed
  by Madness''
  \href{http://io9.com/ive-seen-the-worst-memes-of-my-generation-destroyed-by-464948581}{link}
  (note comment by O'Reilly) about Morozov, E. ``The Meme Hustler''
  \href{http://www.thebaffler.com/past/the_meme_hustler}{link}
\item
  Froomkin, A. Michael. \emph{Read only Section II, pp.~777-795.}
  ``Habermas@Discourse. Net: Toward a Critical Theory of Cyberspace.''
  Harvard Law Review 116, no. 3 (January 2003): 749--873.
  \href{http://osaka.law.miami.edu/~froomkin/discourse/ils.pdf}{link}
\item
  Chapter 3, ``The Movement'', and Chapter 4, ``Sharing Source Code'',
  from Kelty, C. \emph{Two Bits}
  \href{http://twobits.net/pub/Kelty-TwoBits.pdf}{link}
\item
  Hill, B ``Mako''. ``When Free Software isn't (Practically) Better.''
  \href{http://media.libreplanet.org/u/libby/m/mako/}{video}
\end{itemize}

Optional :

\begin{itemize}
\itemsep1pt\parskip0pt\parsep0pt
\item
  Raymond, E. ``The cathedral and the bazaar.'' Knowledge, Technology \&
  Policy 12, 23-49 (1999).
  \href{http://www.catb.org/esr/writings/homesteading/}{link}
\item
  Raymond, E. ``A Brief History of Hackerdom.''
  \href{http://www.catb.org/esr/writings/cathedral-bazaar/hacker-history/}{link}
\item
  Levy, S. Hackers: Heroes of the Computer Revolution (O'Reilly, 2010),
  3rd edn.
\item
  Belenzon, S. \& Schankerman, M. A. Motivation and sorting in open
  source software innovation. Social Science Research Network Working
  Paper Series (2009). \href{http://ssrn.com/abstract=1401776}{working
  paper}.
\end{itemize}

\textbf{10/1 Local Pelican Installation LAB 4}

\textbf{10/4 Game Theory and The Evolution of Cooperation - Thomas}

Required :

\begin{itemize}
\itemsep1pt\parskip0pt\parsep0pt
\item
  Axelrod, R. The Evolution of Cooperation: Revised Edition (Basic
  Books, 2006), revised edn. : Chapter 2 : The Success of TIT FOR TAT in
  Computer Tournaments
  \href{http://jcr.sagepub.com/content/24/1/3.short}{article}, Chapter 3
  : The Chronology of Cooperation
  \href{http://www.jstor.org/stable/1961366}{article}
\item
  Gächter, S., von Krogh, G. \& Haefliger, S. Initiating
  private-collective innovation: The fragility of knowledge sharing.
  Research Policy 39, 893-906 (2010)
  \href{http://www.smi.ethz.ch/news/docs/GachtervonKroghHaefliger_ResPol2010.pdf}{article}
\end{itemize}

Optional :

\begin{itemize}
\itemsep1pt\parskip0pt\parsep0pt
\item
  Axelrod, R. The Evolution of Cooperation: Revised Edition (Basic
  Books, 2006), revised edn. (excerpts) : Chapter 5 :
  \href{http://www.sciencemag.org/content/211/4489/1390.short}{article}
\item
  Helbing, D. \& Yu, W. The outbreak of cooperation among success-driven
  individuals under noisy conditions. Proceedings of the National
  Academy of Sciences 106, 3680-3685 (2009).
\end{itemize}

\textbf{10/8 (Class meet in IRC channel) LAB 5}

\emph{ASSIGNMENT 2: Report on the history, infrastructure, and
demographics of the project.}

\begin{itemize}
\itemsep1pt\parskip0pt\parsep0pt
\item
  \emph{Why is the project open? Is it for ideological reasons, or
  practical reasons, or both?}
\item
  \emph{How big is the community? Where are its members located? How did
  you find out?}
\item
  \emph{What kind of product is it and how is it licensed?}
\item
  \emph{What sort of infrastructure does it use? Why does it use those
  tools and not other options?}
\end{itemize}

\emph{Feel free to ask the community directly about these questions.
There may be historical archived records of conversations about these
decisions. Please provide links to any evidence you use in this report.}

\textbf{10/11 Project Governance - Seb}

Required :

\begin{itemize}
\itemsep1pt\parskip0pt\parsep0pt
\item
  ``Apache Voting Process''
  \href{http://www.apache.org/foundation/voting.html}{link}
\item
  Fogel, K. ``Social and Political Infrastructure''
  \href{http://producingoss.com/en/social-infrastructure.html}{link}
\item
  Freeman, J. The ``Tyranny of Structurelessness''
  \href{http://www.jofreeman.com/joreen/tyranny.htm}{link}
\item
  Shah, S. K. Motivation, governance, and the viability of hybrid forms
  in open source software development. Manage. Sci. 52, 1000-1014
  (2006).
  \href{http://faculty.washington.edu/skshah/Shah\%20-\%20Motivation,\%20Governance,\%20Hybrid\%20Forms.pdf}{link}
\end{itemize}

Optional :

\begin{itemize}
\itemsep1pt\parskip0pt\parsep0pt
\item
  ``Arrows Impossibility Theorem''
  \href{http://en.wikipedia.org/wiki/Arrow\%27s_impossibility_theorem}{link}
\end{itemize}

Example bylaws documents:

\begin{itemize}
\itemsep1pt\parskip0pt\parsep0pt
\item
  Apache Pig Bylaws
  \href{http://www.apache.org/foundation/voting.html}{link}
\item
  GeoServer Community Process
  \href{http://docs.geoserver.org/stable/en/developer/policies/community-process.html}{link}
\item
  Linux Bylaws \href{http://www.linuxfoundation.org/about/bylaws}{link}
\item
  Wikipedia Governance -
  \href{http://p2pfoundation.net/Wikipedia_-_Governance}{link}
\end{itemize}

\textbf{10/15 (Class meet in IRC channel) LAB 6}

\emph{ASSIGNMENT 3: Blog post about community participation. Incorporate
links to your project participation and engage the readings. Do they
generalize to your experience? Or not? How?}

\textbf{10/18 Community Joining and Governance (in relation with the
Class Collective Project)}

Required :

\begin{itemize}
\itemsep1pt\parskip0pt\parsep0pt
\item
  von Krogh, G., Spaeth, S. \& Lakhani, K. R. Community, joining, and
  specialization in open source software innovation: a case study.
  Research Policy 32, 1217-1241 (2003).
  \href{http://flosshub.org/system/files/rp-vonkroghspaethlakhani.pdf}{link}
\item
  Tomlinson, B. et al. Massively distributed authorship of academic
  papers. In CHI '12 Extended Abstracts on Human Factors in Computing
  Systems, CHI EA '12, 11-20 (ACM, New York, NY, USA, 2012).
  \href{http://dx.doi.org/10.1145/2212776.2212779}{link}
\item
  O'Mahony, S. \& Ferraro, F. The emergence of governance in an open
  source community. Academy of Management Journal 50, 1079-1106 (2007).
  \href{http://dx.doi.org/10.5465/amj.2007.27169153}{link}
  \href{http://www.jstor.org/stable/20159914}{alternative link}
\end{itemize}

\textbf{10/22 LAB 7 - Meet in South Hall 202. We will work on the class
survey.}

\emph{ASSIGNMENT 4: How does your community make tough decisions? What
is it's governmance model--for example, is it a benevolent dictatorship,
or consensus driven? How did it get to be that way? Do you think this
governance model is conducive to cooperation on your project? Are there
hidden power dynamics in your project that influence decision-making but
are not explicitly part of the governance model? Think critically about
the social organization of your project: could you improve on it? Where
possible, link to your community's policy documents and examples of
community behavior.}

\textbf{10/25 Money - Seb}

Read:

\begin{itemize}
\itemsep1pt\parskip0pt\parsep0pt
\item
  Asay, M. ``Fitting the optimal level of openness to your business
  strategy''
  \href{http://news.cnet.com/8301-13505_3-10244853-16.html}{link}
\item
  Fogel, K. ``Money'' \href{http://producingoss.com/en/money.html}{link}
\item
  Kelty, C. ``Conceiving Open Systems'', only the sections
  \emph{Hopelessly Plural}, \emph{Open Systems One: Operating Systems},
  \emph{Figuring Out Goes Haywire}, \emph{Denuemont}.
  \href{http://www.jus.uio.no/sisu/two_bits.christopher_kelty/5.html}{link}
\item
  Pentaho. ``The Beekeeper''
  \href{http://wiki.pentaho.com/display/BEEKEEPER/The+Beekeeper}{link}
\item
  Polrid, ``How Does Mozilla Make Money and Stay Afloat?''
  \href{http://www.technobuffalo.com/2010/01/01/how-does-the-mozilla-foundation-make-money/}{link}
\item
  Preston-Werner, T. ``Open Source (Almost) Everything''
  \href{http://tom.preston-werner.com/2011/11/22/open-source-everything.html}{link}
\item
  Worstall, T. ``So Why Is Google Funding Its Own Competition In The
  Firefox OS?''
  \href{http://www.forbes.com/sites/timworstall/2013/01/22/so-why-is-google-funding-its-own-competition-in-the-firefox-os/}{link}
\item
  And don't forget
  Wordpress\ldots{}\href{http://thenextweb.com/insider/2013/03/05/automattic-introduces-wordpress-com-business-a-new-pricing-tier-with-unlimited-themes-storage-and-customer-support/}{link1}
  \href{http://allthingsd.com/20120425/automattic-grows-up-the-company-behind-wordpress-com-shares-revenue-numbers-and-hires-execs/}{link2}
\end{itemize}

Explore:

\begin{itemize}
\itemsep1pt\parskip0pt\parsep0pt
\item
  \href{https://www.bountysource.com/}{BountySource}
\item
  \href{https://www.gittip.com/}{Gittip}
\item
  \href{http://www.kickstarter.com/}{Kickstarter}
\item
  \href{https://snowdrift.coop/}{Snowdrift.coop}
\end{itemize}

\emph{ASSIGNMENT 5: With your classmates, finalize the survey that you
will be using for the class report.}

\textbf{10/29 LAB 8 -- Meet in SH 202 to complete survey and draft
outline of the paper.}

\textbf{11/1 (11am - 12pm) Social Networks, Cooperation, and Group
Performance - Thomas}

Required :

\begin{itemize}
\itemsep1pt\parskip0pt\parsep0pt
\item
  Hanaki, N., Peterhansl, A., Dodds, P. S. \& Watts, D. J. Cooperation
  in evolving social networks. Management Science 53, 1036-1050 (2007)
  \href{http://mansci.journal.informs.org/content/53/7/1036.short}{article}.
\end{itemize}

Optional :

\begin{itemize}
\itemsep1pt\parskip0pt\parsep0pt
\item
  Fu, F., Hauert, C., Nowak, M. A. \& Wang, L. Reputation-based partner
  choice promotes cooperation in social networks. Physical Review E 78,
  026117+ (2008)
  \href{http://dx.doi.org/10.1103/physreve.78.026117}{article}.
\item
  Blundell, C., Heller, K. \& Beck, J. Modelling reciprocating
  relationships with hawkes processes. In Advances in Neural Information
  Processing Systems 25, 2609-2617 (2012)
  \href{www.gatsby.ucl.ac.uk/~ucgtcbl/papers/BluHelBec2012.pdf‎}{article}
\item
  Sparrowe, R. T., Liden, R. C., Wayne, S. J. \& Kraimer, M. L. Social
  networks and the performance of individuals and groups. Academy of
  Management Journal 44, 316-325 (2001)
  \href{http://www.jstor.org/stable/3069458}{article}
\item
  Zimmermann, M. G. \& Egu'iluz, V. M. Cooperation, social networks, and
  the emergence of leadership in a prisoner's dilemma with adaptive
  local interactions. Physical Review E 72, 056118+ (2005)
  \href{http://dx.doi.org/10.1103/physreve.72.056118}{article}.
\item
  Apicella, C. L., Marlowe, F. W., Fowler, J. H. \& Christakis, N. A.
  Social networks and cooperation in hunter-gatherers. Nature 481,
  497-501 (2012) \href{http://dx.doi.org/10.1038/nature10736}{article}
\item
  Woolley, A. W., Chabris, C. F., Pentland, A., Hashmi, N. \& Malone, T.
  W. Evidence for a collective intelligence factor in the performance of
  human groups. Science 330, 686-688 (2010).
  \href{http://dx.doi.org/10.1126/science.1193147}{article}.
\end{itemize}

\textbf{11/1 (12pm - 1am ) Designing Organizations for Productive Bursts
- Thomas} Required :

\begin{itemize}
\itemsep1pt\parskip0pt\parsep0pt
\item
  Georg von Krogh, Thomas Maillart, Stefan Haefliger, Didier Sornette,
  Designing Organizations for Productive Bursts (under review) 2013.
  (manuscript will be sent on the mailing list)
\end{itemize}

Optional :

\begin{itemize}
\itemsep1pt\parskip0pt\parsep0pt
\item
  von Krogh, G., Haefliger, S., Spaeth, S. \& Wallin, M. W. Carrots and
  rainbows: Motivation and social practice in open source software
  development. MIS Quarterly 36 (2012)
  \href{http://search.ebscohost.com/login.aspx?direct=true\&db=bth\&AN=74756698\&site=ehost-live}{paper}
\item
  Benkler, Y. Coase's penguin, or, linux and ``the nature of the firm''.
  The Yale Law Journal 112, 369+ (2002)
  \href{www.yale.edu/yalelj/112/BenklerWEB.pdf‎}{paper}.
\item
  Maillart et al., Superlinear Productivity in Open Source Software
  (manuscript will be sent on the mailing list)
\end{itemize}

\textbf{11/5 LAB 9 ( Survey Validation and Launch )}

\textbf{11/8 International collaboration / Practicum Updates - Seb}

Readings:

\begin{itemize}
\itemsep1pt\parskip0pt\parsep0pt
\item
  Dravis, P. ``Open Source Software: Perspectives for Development''
  \href{http://www.infodev.org/infodev-files/resource/InfodevDocuments_21.pdf}{link}
  Parts I and II.
\item
  Takhteyev, Y. and Hilts, A. ``Investigating the Geography of Open
  Source Software through Github''
  \href{http://www.takhteyev.org/papers/Takhteyev-Hilts-2010.pdf}{link}
\item
  Takhteyev, Y. pp.~94-147 (105-158 in the linkedPDF) ``1.2 The Global
  Tongue'' in \textbf{Coding Places: Uneven Globalization of Software
  Work in Rio de Janeiro, Brazil}
  \href{http://codingplaces.net/static/takhteyev_2009_dissertation.pdf}{link}
\end{itemize}

\textbf{11/12 LAB 10 ( Survey Analysis )}

\textbf{11/15 (11am - 12pm) Economics and Management of Modularity -
Thomas}

Required :

\begin{itemize}
\itemsep1pt\parskip0pt\parsep0pt
\item
  Baldwin, C. Y. \& Clark, K. B. The architecture of participation: Does
  code architecture mitigate free riding in the open source development
  model? Manage. Sci. 52, 1116-1127 (2006)
  \href{http://mansci.journal.informs.org/content/52/7/1116.abstract}{paper}
\end{itemize}

Optional : - MacCormack, A., Rusnak, J. \& Baldwin, C. Y. Exploring the
structure of complex software designs: An empirical study of open source
and proprietary code. Management Science 52, 1015-1030 (2006)
\href{http://pubsonline.informs.org/doi/abs/10.1287/mnsc.1060.0552}{paper}.
- MacCormack, A., Baldwin, C. \& Rusnak, J. Exploring the duality
between product and organizational architectures: A test of the
``mirroring'' hypothesis. Research Policy 41, 1309-1324 (2012)
\href{http://dx.doi.org/10.1016/j.respol.2012.04.011}{paper}. -
Maillart, T., Sornette, D., Spaeth, S. \& von Krogh, G. Empirical tests
of zipf's law mechanism in open source linux distribution. Physical
Review Letters 101, 218701+ (2008)
\href{http://prl.aps.org/abstract/PRL/v101/i21/e218701}{paper}. - Simon,
H. The architecture of complexity. Proceeding of American Philosophy
Society 106, 467-482 (1962)
\href{http://www.jstor.org/discover/10.2307/985254}{paper}. - Gal, M. S.
Viral open source: Competition vs.~synergy. Journal of Competition Law
and Economics (2012)
\href{http://papers.ssrn.com/sol3/papers.cfm?abstract_id=1978374}{paper}

\textbf{11/15 (12pm - 1pm) Guest Lecture - Civic Hacking {[}Sophia
Parafina{]}}

\textbf{11/19 LAB 11 ( Collective Report Writing )}

\textbf{11/22 (11am - 12pm) Human Timing and Economics of Attention -
Thomas}

Required :

\begin{itemize}
\itemsep1pt\parskip0pt\parsep0pt
\item
  Maillart, T., Sornette, D., Frei, S., Duebendorfer, T. \& Saichev, A.
  Quantification of deviations from rationality with heavy tails in
  human dynamics. Physical Review E 83, 056101+ (2011).
  \href{http://arxiv.org/abs/1007.4104}{preprint}
\end{itemize}

Optional :

\begin{itemize}
\itemsep1pt\parskip0pt\parsep0pt
\item
  Barabási, A.-L. The origin of bursts and heavy tails in human
  dynamics. Nature 435, 207-211 (2005). URL
  http://dx.doi.org/10.1038/nature03459.
\end{itemize}

\textbf{11/22 (12pm - 1pm) Guest Lecture - Project Teams in Community
and Commercial Open Source Software Projects -
\href{http://about.me/tony.wasserman/bio}{Tony Wasserman} {[}Carnegie
Mellon Silicon Valley{]}}

This talk describes the recent evolution of business strategies used by
companies offering products and services based on free and open source
software (FOSS). The talk compares their practices with traditional
proprietary software companies and with community-based open source
projects, and identifies growing overlaps between the different kinds of
software companies. We focus on the similarities and differences among
development teams on various open source projects, describing the
variations in project governance and the nature of communities in both
commercial and community open source projects, including the issues that
lead to success or failure of such projects.

\textbf{11/26 LAB 12 ( Collective Report, First Complete Draft )}

\textbf{12/03 LAB 13 (collective report writing , if needed)}

\emph{ASSIGNMENT 6: Where does the funding for you community come from?
Is there corporate sponsorship? A foundation that backs it? Do users
donate? How does this affect the community's cooperative dynamics? Are
there competing projects? How would you describe your project's role in
the greater technical ecosystem?}

\textbf{12/6 (11am - 12pm) Guest Facilitator - Gender and Open Source
{[}Rochelle Terman{]}}

Readings:

\begin{itemize}
\itemsep1pt\parskip0pt\parsep0pt
\item
  TBD
\end{itemize}

\textbf{12/6 Open Collaboration and Education; Class Wrap-up; Course
Evaluations - Seb}

Readings:

\begin{itemize}
\itemsep1pt\parskip0pt\parsep0pt
\item
  Ellis, H.J.C., Morelli, R., de Lanerolle, T., Damon, J. and Raye, J.
  Can humanitarian open-source software development draw new students to
  CS? In Proceedings of the 38th SIGCSE Technical Symposium on Computer
  Science Education (Covington, Kentucky, USA), March 2007, pp.~551-555.
  \href{http://www.cs.trincoll.edu/hfoss/images/0/0a/Ellis_etal_SIGCSE_2007.pdf}{pdf}
\item
  Practical OSS Exploration. Forward.
  \href{http://teachingopensource.org/index.php?title=Practical_OSS_Exploration_-_Foreword\&direction=next\&oldid=3605}{link}
\item
  \href{http://www.hfoss.org/index.php/publications-etc}{see also}
\end{itemize}

\textbf{12/10 LAB 14 (for collective report writing, optional)}

\emph{ASSIGNMENT 7: Does your project's community mirror the technical
modularity of the project? How does it structure its
collaboration--synchronously? Asyncronously? How does it get work done?}

\textbf{12/13 Optional meeting in SH 210 for report writing}

\textbf{12/16 Final blog post revisions due }

\textbf{12/17 LAB 15 (collective report writing, if needed)}

\textbf{12/20 Collective Report due, if not finished earlier (no class)}
