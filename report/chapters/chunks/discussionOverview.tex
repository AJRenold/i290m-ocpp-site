In the begining of the class we read a paper \cite{tomlinson2012} in which 30 authors wrote a paper about the process of writing a paper with 30 authors. This paper was largely an experiment of sorts surrounding things we had been learning in class, it was a chance for us to test out new theories of collaborative work.

We think much of that needs to be questioned. There is most definitely room for improvement. This is the place for that discussion; the meta-analysis.

Some interesting questions:
Would this report have been successfully written if we weren't taking this class for a grade?
Have we produced anything of value?
What tools would have made this project easier? 
How do you write with so many other people? 
How can you achieve a consistent tone and style? 

However, we wonder, how can we say anything? Many of us have different experiences that lead us to disagree with each other, and in a system where we are all weighted equally... how can we make a decision on what to write?

We wonder what a paper with 30 equal authors making equal contributions really means. Is it the voice of 30 people if only 2 of the 30 people have read the whole paper?

Surprisingly, it wasn't always hopeless. We did get work done if we could keep people interested in the project, and even though we we're leaderless, leaders eventually emerged. In many ways this was the best kind of experiment: learning by doing.  


\mysubsubsection{The lack of structure}
I'm lost. I can't see what others are writing and I have no perspective on how this paragraph will fit into the larger whole. Will it just fit? Will it be meaningful? Will it exist in a flow of thought? All of this makes me wonder how far we need to come with collaborative writing and coding tools. There needs to be a better vantage point of what others are doing and what needs to be done. GitHub provides some of this, but it's not in real time. Google docs is in real time, but it's uncontrolled chaos if 30 people are writing. What would a system in the middle look like?

I am finding the process surrounding the production of this report quite challenging and confusing. 
There's an argument to be made that this is what participating in an open-source project feels like. 
It's just part of the package. And true, that is often the case. But there are some important
differences between an open-source software or editorial project and this report.
First of all, prose writing is in general much less easily modularized than code. 
Other editorial projects make up for this with tools well suited to the job and/or
not trying to write one large document that is coherent across its sections. But in this 
case we did neither. I feel too siloed from my co-authors to make what feels like a helpful 
contribution---the tools and process that the instructors chose for us prevent you from seeing what others 
are doing in anything approaching real time. And the chaos of this process compared to other editorial processes I have 
been a part of is, in my opinion, ill-suited to a high-quality end product. Someone who 
understands editing and the editorial coordination process needs to be steering the ship for 
a collaboration of 20+ writers to work. However, upon thinking it through more, there's another aspect of the problem that 
feels even more important: we've been trying to treat this report like an open-source project, 
but the activities we're undertaking are only partly motivated the same way as open-source 
participation. We've chosen to take the class, but we are being graded/given course credit 
for our activities. n a regular open-source project, one can withdraw when things become 
confusing or chaotic, or only commit to things that feel do-able in the requested timeframe. 
In class, though, if you want to be sure of passing, you can't reasonably opt out of the 
required work. There is a level of coercion inherent in the institution. (I mean literally, 
in the institution. Pressure does not come the instructors. Our instructors have to grade 
us even if they would prefer not to.) My recommendation to address this problem in future 
iterations of the class is to change the grading rubric so that students can have more freedom 
to choose how they participate in class activities. Rather than having a set percentage of 
a grade derived from the each of report, the blog posts, and contribution to an outside project, 
percentages should be more flexible based on where a student chooses to put her efforts. 
The areas of contribution should be: contribution to an open collaboration community, reporting on 
the experience (the blog posts), and producing the collaborative report; each student should be 
required to participate in two out of the three but not all. Furthermore, the class itself 
should be considered an open collaboration community that could be joined. Contributions 
could include actively hashing out governance issues, facilitating discussions, finding 
helpful readings on topics that come up but are not provided for in the syllabus, and taking on 
community management tasks that are a part of any well-run open collaboration project. 

\subsubsection{Discussion on Process}

One of the biggest hurdles that seem to come about the process of working on this survey and corresponding report primarily involves organization and clarity on responsibilities and roles. When having so many poeple contribute to a survey at once without any sort of restraint or control, there seems to have been a lot of redundant questions or questions that overlap with so many others. It was good that we ultimately chose someone to organize the content, but the process and the hurdles felt prolonged and slighly unorganized. Ultimately this does feel like it reflects an open source organization in the fact that there are mass amounts of contributions that required one benevolent dictator to organize and maintain quality control of the content. If it weren't for someone's intervention to organize the survey, it would have been much more of a chore to fill out and analyze. 

The final result of the survey came about easily enough. Google Survey assisted in making the results easy to digest and straightforward to discuss in a report. However, the discussion and exercises to understand what to do with this data in a report involved more discussion in a democratic way that once again, felt a little slow and a bit muddling. So many ideas and proposals on how to go about writing this report came up that it became easy to get lost about what direction everyone is actually taking. Once again I feel like this came about because there was a high emphasis on trying to get everyones input and thoughts in, leading to what felt like a really circular discussion. While I see the value in trying to get everyone's thoughts in the process, I felt as if there should be a stronger leading figure to ease along the final decision making process. 

The exercises in writing the modular parts of this paper were useful though, as well as the ultimate structure of how this report is written. Breaking off into groups to write one large paper, with each smaller group having a leadership figure to oversee the organization and writing process. This seems especially effective due to the fact that one large report can now be written with smaller parts contributed by everyone, and then organized by those responsible for each chapter. The only flaw I see in this is that it may possibly lead to a large paper that has stylistic inconsistencies or unorganized thoughts between chapters. The responsibility of going over the entire report and cleaning the content is a large role and hopefully can be done with ease. 

\subsection{Limitations}
\label{sec:limitations}

Our survey sample is small ($n=24$), and therefore the statistical analysis should be considered more indicatory than rigorous. As several students have participated in the same projects, those projects over represented in our data. This implies our data is skewed towards these projects, especially peerlibrary (n=5) and hypothes.is (n=3).

After reviewing the results in the class, the survey was found to be inadequate in several ways. First, it was biased towards software development, even though not all respondents were involved in software development projects. Secondly, some questions were found difficult, such as the licensing question (\#nn), which left out important licenses such as the Apache license, and number of female contributors (\#nn), which was identified as difficult to answer. We attribute these problems to the rather chaotic process of contributing and lack of quality control.  Many of these problems were already discussed in the previous efforts of academic collaboration \cite{Tomlinson2012}. I suggest that in future, instead of circa 20 participants editing the survey, a benevolent dictator be chosen to coordinate the collaboration effort.


