\mysubsubsection{Meta-Analysis}\\
In the begining of the class we read a paper \cite{tomlinson2012} in which 30 authors wrote a paper about the process of writing a paper with 30 authors. This paper was largely an experiment of sorts surrounding things we had been learning in class, it was a chance for us to test out new theories of collaborative work.

I think much of that needs to be questioned. There is most definitely room for improvement. This is the place for that discussion; the meta-analysis.

Some interesting questions:
Would this report have been successfully written if we weren't taking this class for a grade?
Have we produced anything of value?
What tools would have made this project easier? 
How do you write with so many other people? 
How can you achieve a consistent tone and style? 

However, I/we wonder, how can we say anything? Many of us have different experiences that lead us to disagree with each other, and in a system where we are all weighted equally... how can we make a decision on what to write?

Speaking for myself (am I allowed to do this?), I wonder what a paper with 30 equal authors making equal contributions really means. Is it the voice of 30 people if only 2 of the 30 people have read the whole paper?

If I put my name on it is it an endorsement of everything that has been said, or is it simply tied to my contributions? Will I be allowed to have a controversial opinion or will it be edited out of the final report? We shall see.

Surprisingly, it wasn't always hopeless. We did get work done if we could keep people interested in the project, and even though we we're leaderless, leaders eventually emerged. In many ways this was the best kind of experiment: learning by doing.  

This is the discussion section, a place for people who have been part of this experiment to voice their experience of writing this paper, disagreements, and more. Each subsection will contain the thoughts of one contributor to the process. They have been encouraged to be as honest as possible.




