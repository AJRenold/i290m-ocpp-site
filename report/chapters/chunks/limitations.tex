\subsection{Limitations}
\label{sec:limitations}

Our survey sample is small ($n=24$), and therefore the statistical analysis should be considered more indicatory than rigorous. As several students have participated in the same projects, those projects over represented in our data. This implies our data is skewed towards these projects, especially peerlibrary (n=5) and hypothes.is (n=3).

After reviewing the results in the class, the survey was found to be inadequate in several ways. First, it was biased towards software development, even though not all respondents were involved in software development projects. Secondly, some questions were found difficult, such as the licensing question (\#nn), which left out important licenses such as the Apache license, and number of female contributors (\#nn), which was identified as difficult to answer. We attribute these problems to the rather chaotic process of contributing and lack of quality control.  Many of these problems were already discussed in the previous efforts of academic collaboration \cite{Tomlinson2012}. I suggest that in future, instead of circa 20 participants editing the survey, a benevolent dictator be chosen to coordinate the collaboration effort.