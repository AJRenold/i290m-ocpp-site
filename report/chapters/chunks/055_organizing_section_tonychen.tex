\subsubsection{Organizations}

The organizational structure of open source projects is dynamic and latent. Many projects are effectively self-organized and there is a significant degree of variation in the organization and governance of projects that can depend on project size and contributor composition. It is worthwhile to investigate how open source projects self-organize by examining trends in collaboration behavior across complex open source communities.

The organizational structure of a project needs to facilitate an atmosphere for consistent progress of development. Initially, a self-organized group of developers seek out to solve a problem and typically the project's founding developer or developers are responsible for shaping the initial structure and recruiting new contributors to the project. The roles of the initial members are relatively general and this core group is bound by a very informal, ad-hoc structure. New contributors who believe in the vision of the project or derive value from the project join the community in a variety of ways. For example, a self-motivated joiner can enter a community by doing small tasks like editing documentation and gradually integrate into the developer community by taking on larger tasks. Eventually, this joiner can become a part of the community's core group by developing expertise on the project or a module of the project. As this process takes place with more and more joiners, a project will gradually develop a unique culture and organization needed to facilitate the completion of work among contributors. There is often no direct assignment of work by a leader or core team, but the open nature of the project and its infrastructure creates a visibility of work that is important to all community participation. An individual leader or core team's influence is also important in shaping the culture and direction of the project, but their roles can be more facilitative than authoritative.
