\subsubsection{Visibility of Work in Open Organizations}

In an open source project, whether software or non-software, infrastructure has made it possible for contributions to be visible to the community in some way. Software projects are supported by version control and issue or bug tracking and non-software projects such as Wikipedia are supported by wiki technology which essentially provides version control for documents. Both types of projects are often supported by additional infrastructure such as IRC chats, mailing lists, discussion forums and blogs. An important affordance of the infrastructure supporting the work in a project is that the infrastructure typically makes work units visible to other community members. Examining Github for example, reveals a significant number of features that attach ownership of work units to project participants. Github's Issues feature, where the issues themselves can be considered a unit of work, allows one of more contributor to be assigned to an Issue. Similarly a Pull Request on Github is attributed to the author and the author remains visible in the code through Git's 'blame' feature, where lines of code can be annotated by contributor. Looking at a wiki such as Wikipedia, the units of work are the actual contents of each wiki and are therefore always visible. Wiki technology also allows for changes made to a page to be attributed to a specific author. While it is not a revelation that open source projects make visible the work of community members, it is worth examining the benefits to the organization and distribution of work tasks, and the community structure that arises out of openness.

{\it Ownership} - One important effect of visible work in open source projects is the ownership of work before and after completion. Before a task is completed, it is highly useful for the task to be assigned to or claimed by a community contributor. First, the visibility of ownership of a work unit prevents other community participants from unnecessarily duplicating efforts to complete the task. Reflecting on participating in authoring parts of this paper, little writing was completed before contributors were assigned to work on specific sections of the paper. When contributors knew which section or subsection they would author, there was further formation of subgroups and then delegation of tasks to subgroup members (At least for this section). Second, the visibility of ownership of work can be a motivator for many contributors to participate. For example, a Github user name is now a common feature on a software developers resume or and is often requested by a potential employer. Further supporting this, 7 of the 23 survey respondents claimed ``Career Advancement" as a motivation for their open source contributions. This suggests that participants derive value from the visibility of work ownership in an open source project because participation could lead toward improved job prospects with a future employer.

{\it Coordination and Delegation} - The next important effect of visible work in open source projects is the coordination of work between a wide range of project contributors. Where ownership allows contributors of a project to avoid the duplication of work, the visibility of tasks without ownership enables coordination and delegation of work. Even for a new community participant who joins simply to report a bug or request a feature, the visibility of this request, whether in a community mailing list or in a issue tracker represents the creation of an unclaimed work unit that has been delegated to the community. If the bug is a major bug, or if the request is popular, a community contributor might claim the task or be assigned to the task by a community leader. The creation of the bug and the claim by a contributor is both a delegation and coordination activity. This also occurs in a longer time frame through the creation of more formal community documents such as a development roadmap or enhancement proposals, and more informal avenues such as a mailing list. In the longer term, the creation of these documents or discussions allows the more involved community contributors, often called the 'core' contributors, to develop and agree upon a shared understanding of the direction of the project. This might involve the use or creation of community governance structures such as voting rules or the creation of benevolent dictators (or other leadership of some kind).

Ultimately, the visibility of work units or tasks in peer produced project has a significant impact on the end result of the project. Without infrastructure supporting the visibility of work, the ownership, coordination, delegation and completion of work would be difficult if not impossible. For one, it would be unimaginable to be an author of this paper without knowledge of the units of work that have been assigned to or claimed by community participants.
