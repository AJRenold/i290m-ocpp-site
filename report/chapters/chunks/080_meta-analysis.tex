\mysubsubsection{Stream of conscious musings on the writing of this report}
I'm lost. I can't see what others are writing and I have no perspective on how this paragraph will fit into the larger whole. Will it just fit? Will it be meaningful? Will it exist in a flow of thought? All of this makes me wonder how far we need to come with collaborative writing and coding tools. There needs to be a better vantage point of what others are doing and what needs to be done. GitHub provides some of this, but it's not in real time. Google docs is in real time, but it's uncontrolled chaos if 30 people are writing. What would a system in the middle look like?

I am finding the process surrounding the production of this report quite challenging and confusing. 
There's an argument to be made that this is what participating in an open-source project feels like. 
It's just part of the package. And true, that is often the case. But there are some important
differences between an open-source software or editorial project and this report.
First of all, prose writing is in general much less easily modularized than code. 
Other editorial projects make up for this with tools well suited to the job and/or
not trying to write one large document that is coherent across its sections. But in this 
case we did neither. I feel too siloed from my co-authors to make what feels like a helpful 
contribution---the tools and process that the instructors chose for us prevent you from seeing what others 
are doing in anything approaching real time. And the chaos of this process compared to other editorial processes I have 
been a part of is, in my opinion, ill-suited to a high-quality end product. Someone who 
understands editing and the editorial coordination process needs to be steering the ship for 
a collaboration of 20+ writers to work. However, upon thinking it through more, there's another aspect of the problem that 
feels even more important: we've been trying to treat this report like an open-source project, 
but the activities we're undertaking are only partly motivated the same way as open-source 
participation. We've chosen to take the class, but we are being graded/given course credit 
for our activities. n a regular open-source project, one can withdraw when things become 
confusing or chaotic, or only commit to things that feel do-able in the requested timeframe. 
In class, though, if you want to be sure of passing, you can't reasonably opt out of the 
required work. There is a level of coercion inherent in the institution. (I mean literally, 
in the institution. Pressure does not come the instructors. Our instructors have to grade 
us even if they would prefer not to.) My recommendation to address this problem in future 
iterations of the class is to change the grading rubric so that students can have more freedom 
to choose how they participate in class activities. Rather than having a set percentage of 
a grade derived from the each of report, the blog posts, and contribution to an outside project, 
percentages should be more flexible based on where a student chooses to put her efforts. 
The areas of contribution should be: contribution to an open collaboration community, reporting on 
the experience (the blog posts), and producing the collaborative report; each student should be 
required to participate in two out of the three but not all. Furthermore, the class itself 
should be considered an open collaboration community that could be joined. Contributions 
could include actively hashing out governance issues, facilitating discussions, finding 
helpful readings on topics that come up but are not provided for in the syllabus, and taking on 
community management tasks that are a part of any well-run open collaboration project. 
