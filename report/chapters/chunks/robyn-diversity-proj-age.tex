
{\bf Question about Project Founding Date}
We asked the question about project founding date to consider relative ages of different FLOSS communities. This set of ages reported helps to identify different stages of FLOSS project development. We expected that there might be patterns in communities as they are born, grow, reach maturation, and persist over time, or fizzle out. 

Among the projects surveyed, all started after 2001, and nearly half sprang up during or after 2011. As small a range as this is, we can consider the projects according to four ages: new (born in or after 2011), settling down (between 2008-2010), established (between 2007-2005), and anything older old. (If surveying a larger community, the ranges might be collapsed differently to account a more uniform spread across a larger range of years.)

{\bf Diversity Questions}
This set of questions aims to gather information on various kinds of diversity within projects. As has been cited in various surveys with respect to gender diversity, FLOSS projects suffer from a dearth of women contributors. \footnote{See a number of sources on women in FLOSS communities here: http://geekfeminism.wikia.com/wiki/Open_Source_Software} We expect to see similar patterns in other demographic groups that tend to be underrepresented elsewhere in society.

Metrics related to diversity are challenging to define, and answers will reflect biases of answerers about definitions of diversity and participation (with whom) in the project and community. Nonetheless, we expect responses to shade in our understanding about the demographics of different projects, and hopefully correlate with other community characteristics. Ultimately, we are curious about the relationship between diversity and how participation plays out. The additional question will hopefully capture more broad statements and intuitions about diversity (and perhaps other types of diversity) in the community.

According to respondents, project communities are made up of people of diverse backgrounds and experiences on all levels except the socioeconomic level. What we can gather from the write-in answers, however, is that respondents experienced some difficulty in assessing the demographics of all the communities due to remote communication and lack of demographic data collection. 

Still, comments from those who were able to report on demographics suggest a trend that FLOSS project leadership and contributors are mostly middle- to upperclass males who are mostly from developed countries. 

These questions on diversity and their responses necessitate deeper study in the future to better understand the demographic make-up across a wider range of FLOSS communities and how such diversity affects communication, movement, innovation in projects. 


