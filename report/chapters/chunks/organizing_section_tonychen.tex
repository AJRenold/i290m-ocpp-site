\subsubsection{Organization Tony Chen}

The organizational structure (if a structure did exist) of open source projects is dynamic and latent. There does not seem to be a universal pre-designed organizational structure for any given project due to its self-organizing nature. Many projects are effectively self-organized, but there is a significant degree of variation in the structure and control across different projects that depend on their size and contributor composition. It is, then, worthwhile to investigate how open source projects self-organize by extracting trends in collaboration behavior across complex open source communities.

The organizational structure needs to facilitate a atmosphere for consistent progress of development for the project. Initially a self-organized group of developers seek out to solve a problem with open source and typically the projects founding developer or developers are responsible for shaping the initial structure and recruit new contributors to the project. The roles of the initial members are relatively general and this core group is bound by a very informal, ad-hoc structure. New contributors who believe in the vision for the organization and are self-motivated can enter a community by doing small tasks like editing documentation and gradually integrate into the developer community by taking on larger subprojects and patches. In this typical structure, there is no direct assignment of work by a leader or core team. Although their influence is important in shaping the culture, their role is more facilitative than authoritative.

In the self-organizing nature of open source projects, visibility of work is important to coordinate a way for members to contribute in anyway possible. An infrastructure around the project's development is crucial in supporting this visibility. As a result, project coordinators use a plethora of tools to provide this support. This includes version control tools and modes in which members can communicate with each other. In software, Github is a popular version control and issue tracking tool used as a way of making work visible to others. It is also a tool for creating documentation for the project. This creates an ecosystem of different ways a team member can contribute to the project whether it is in filing a bug report, writing documentation, providing technical knowledge or editing the source code.
