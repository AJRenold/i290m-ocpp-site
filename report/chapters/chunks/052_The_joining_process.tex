\mysubsubsection{The joining process}
Based on reports from students who joined a host of different kinds of open source projects for the Fall 2013 semester, we can begin to get a glimpse of the onboarding process and the various forms it takes across projects. Some wrote an introductory email \cite{Pham-09-17} and were accepted, while others had to sign a contract \cite{McConachie-10-08}. The structure of the community seems to be important here. Several students point out that a clear infrastructure made it easier to join, since it guides through the joining process and explains how to contribute. Clear and thorough documentation was another factor, however some reported that missing documentation was the initial trigger for starting to contribute. In general, individuals joining open source projects experience a spectrum of responses from warm and welcoming to none at all. Often this depends on the {\it joining script} they use \cite{vonKrogh2013}. The following is a summary of the range of responses students received. 

Most first contacts that got through were very personal. While some responses were extremely helpful and welcoming, a number of projects didn't respond to first-contact attempts. Whether the non-responses were a result of potential participants attempting to join in the ``wrong'' way or through the wrong channel,  projects trying to attract new contributors would do well to make the path to contributing easy and clear. A number of students reported coming across neglected project documentation infrastructure, like unused mailing lists.  A related obstacle to entry for beginners is that convoluted infrastructure can be overwhelming for beginners. Even where projects have explanatory infrastructure for {\it newbies}, the proliferation of too many communication channels sometimes pushed potential contributors to find a human and reach out directly. In the words of the students, this is what the experience was like:

\begin{quotation}
``I scanned through the archive, which goes back to roughly 2011, and it doesn't seem to be heavily used. Many posts for help or feature requests are of poor quality and go unanswered while most bug reporters are pointed to Github issues. There isn't much development discussion occurring on the list.''\\�

``...the forums, the wiki, the mailing lists, and the issue tracker are all separate systems with their own login info.''\\

``...the publicly visible development channels are quiet. A foundation forum is accessible to founding members, but not to normal contributors. A wiki is open to developers, but little discussion takes place here, and shockingly few edit wars.''\\
\end{quotation}

Another factor that has been reported is the prior knowledge of the open source project contributors. The importance of personal connection is discussed more in detail in the following section. It is observed from the survey that if contributors were known as interesting people it had a positive impact on the project.
