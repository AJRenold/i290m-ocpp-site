\subsubsection{Prevalence of the Benevolent Dictatorship Governance Model}

The vast majority of respondents said their project has a benevolent dictator model for governance.  A full 50\% said their governance model was 'benevolent dictatorship' whereas the next most popular choice of 'self-organized' was only chosen by 12\% of respondents.  Given that most of the projects joined by respondents were established open source projects one can then draw the conclusion that benevolent dictatorships are a popular and effective means of getting things done in the open source community.  Were this not the case then many of these projects run by benevolent dictatorships would not have survived to become established projects.

{\it Benefits of Benevolent Dictatorships} The benevolent dictatorship model offers many benefits to peer models of production.  One of the obvious benefits is ease and speed of decision making.  When discussing governance models of their projects respondents emphasized that if their project had a benevolent dictator it would almost always be a person, or a small group of people, whose knowledge of the project exceeded the average contributor's knowledge of the project.  So that the current dictator is usually someone who has been with the project the longest or made the most contributions.  This statement from a respondent presents a great example of the link between project knowledge and decision making power, ``Elaine and Peter in New Zealand are extremely active and know almost everything there is to know about Civi. So is Jamie in NYC and Joe Murray in Canada. These folks have a lot of sway among core team members."  The `sway' that these contributors exhibit is entirely based on their level of activity and knowledge.  Another great example is this one, ``Saracen is the benevolent dictator for life. Gavin Andersson[sic] has a lot of sway as well. most pull requests are yay/nayed ultimately by them."  Both Saracen and Gavin Andresen are heavy contributors to Bitcoin related projects and have proven their worth to the Bitcoin community.

{\it Tyranny of Structurelessness} Many of the respondents polled who did not indicate a dictatorship model as their governance structure indicated that their projects gain structure from corporate or university involvement.  For example, governance in citizen science projects varies largely due to the size of the project. In cases of very large citizen science initiatives like Zooniverse, decisions are made from a top-down approach. Zooniverse employees or university partners make platform and organizational decisions (such as which projects will be hosted) and how the platform will look for each project. The scientific teams (those who open up their projects to be hosted on Zooniverse) make most scientific decisions on the overall project design, formation of research questions, and lead the scientific analysis once contributions are made by citizen scientists. Recently, Zooniverse has experimented by opening up the data analysis portion of a project called “Galaxy Zoo Quench” which attempts to crowd-source more aspects of a hosted project. So Zooiverse avoids the tyranny of structurelessness by piggybacking on the structure provided by universities and their associated university culture.

