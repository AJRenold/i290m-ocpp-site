\subsection{Funding Sources, Business Models \& Revenue Streams}

Within a sample size of 24, a high degree of variance in the types of funding models of open source projects has been observed from analysis of the survey results. 22\% of the responses indicate that funding was received from a for-profit source such as through venture capital investment (7\%) or a direct funding from a for-profit corporation (15\%). On the contrary, a large subset of the surveyed indicated that their funding came from non-profit sources. 24\% of the respondents indicated their project was backed by academic or research grants, 22\% responded that the funding came from donation sources and 17\% responded that the project was financially backed by the members of the core team. The remaining 15\% of the responses indicated that the projects were funded through a crowdfunding campaign such as Kickstarter or Indiegogo.

\mysubsubsection{Effect on License choice} 

We explored the connection between project type and licence in two different ways, in Table \ref{tab:licence_per_fudning} by funding sources and in Table \ref{tab:licence_per_project_type} by project origination. Both tables yield statistically significant results, presented inline in Table \ref{tab:licence_per_fudning} and for Table\ref{tab:licence_per_project_type} $\chi^2$ test gives $p=0.023$.

Based on this analysis we conclude that open source projects with corporate funding prefer the Apache and BSD licenses, while only one project that choose a GNU license received any corporate funding. These results mirror the suggestion of the Mozilla legal team for all Mozilla projects to move over to Apache 2, as it provided some legal patent protections that the other licenses did not. \footnote{\url{https://groups.google.com/forum/#!topic/mozilla.legal/CrwdQkJRfEM}}

\begin{table}[htbp]
  \centering
  \caption{License per funding source}
    \begin{tabular}{|c|c|c|c|c|c|c|}
    \hline
         Funding & AGPL  & BSD   & Apache & MIT   & GNU GPL & p value \\
\hline
    Crowdsourced & 1     & 3     & 1     & 0     & 1     & n.s. \\
    Federal grants & 0     & 1     & 0     & 0     & 0     & n.s. \\
    Non-profit grants & 1     & 4     & 0     & 0     & 2     & p = 0.115 \\
    Not needed or in-kind & 0     & 0     & 0     & 1     & 4     & p = 0.81 * \\
    Corporate / private & 0     & 3     & 3     & 0     & 1     & p = 0.044 ** \\
\hline    
    \end{tabular}
  \label{tab:licence_per_fudning}
\end{table}

\begin{table}[htbp]
  \centering
  \caption{License in different project types}
    \begin{tabular}{|c|c|c|c|c|c|}
	\hline
          License & Corporation lead & Community originating & Academic & nonprofit sector originating \\
    \hline
    AGPL Count & 0     & 0     & 0     & 1 \\
    BSD  Count & 1     & 1     & 2     & 1 \\
    Apache Count & 3     & 0     & 0     & 0 \\
    MIT  Count & 0     & 1     & 0     & 0 \\
    GPL + GNU Count & 0     & 3     & 4     & 0 \\
    \hline
    \end{tabular}
  \label{tab:licence_per_project_type}
\end{table}%


\mysubsubsection{Response Tone and Funding Source}
By separating out the survey responses on funding, who responded and response tone we can see several interesting trends. As shown above, the responses had a high degree of variance but for this comparison, those can be simplified to a binary attribute of funded / unfunded. The responses to "How did the response read" formed a binary attribute as well with all participants selecting either Peer or Teacher.

Though there isn't much data to go by, in funded organizations 63\% of people reported the response tone to their initial contribution read as being answered from a peer rather than as from a teacher. Further exploration shows that unlike unfunded projects, in paid organizations it was rarely the project founder who responded but rather a senior project member. This might explain the presence of a more peer-to-peer response tone verse a pedagogical response tone.

\mysubsubsection{Citizen Science Funding}

Volunteers, or citizen scientists, are not compensated for their observations and contributions to a project. Motivations to contribute to projects may be out of scientific curiosity, a desire to donate time (or in some cases computer resources) to the betterment of society through scientific research, educational gain by working on the project, potential professional development, or occasionally prize money (though this is not the case for Zooniverse projects).

Citizen Science Alliance (CSA) functions as the governing body of the Zooniverse collection of projects. Individual Zooniverse projects are proposed by a scientific team from a university or research institute and then selected for hosting on the Zooniverse platform. The scientific research (and thus the scientific team) are usually funded from Federal or non-profit grants awarded for specific scientific research projects. To be hosted on the Zooniverse platform, projects either bootstrap their own development through their preexisting grants or CSA uses independently acquired grants (e.g. from the Alfred P. Sloan Foundation) to create the citizen science platform.

CSA, or Zooniverse, employees are actually employed by the separate partner institutions like the University of Oxford (the original site) and the Adler Planetarium. Grants to fund the administrative support and technical development are obtained through these partner institutions from Federal and non-profit sources, and then Zooniverse "employees" are employed by the research institutions themselves, but devote their time toward Zooniverse initiatives.

Some citizen science organizations may have a more centralized organization and often the technical team (people who create an observation platform, if needed) and scientific team are integrated and work directly with the volunteers. The consortium of scientific collaborators and projects around Zooniverse makes this centralized organization and funding stream more challenging, so the organization has adapted to make a flexible, cloud-like organization model to float funding and personnel resources between partners.

\mysubsubsection{Individual funding}

The preceding sections discussed funding for projects as a whole, but that only partially represents the realities of funding an open source project. Projects are made up of many different contributors, who have varied ages, jobs, skills and levels of commitment. Our study asked respondents to answer if and how individual contributors are paid to analyze how these variations affected the distributions of funds to individuals. 

Respondents could answer as many choices as applied to the payment sources for the contributors to their project. According to the 23 responses, 35\% of projects had only unpaid participants, while 42\% of the projects surveyed included both paid and unpaid contributors. A mere 13\% of projects had only paid contributors.

The vast majority of projects had contributors that received some funds for their work on the project. Unsurprisingly, all but one "Corporation lead" projects included contributors that were paid as employees of a corporation. Most of those projects also included unpaid contributors. "Community originating projects" were the most varied, getting money from employees, grants, donations and crowd funding. One respondent commented that their projects “core team are full-time employees and sometimes they contract out some work but other than that it's unpaid/"

Finally participants to academic open source projects tended to be unpaid or funded by grants. Likely many contributors to these projects are students who are unpaid but affiliated with the project through professors or classes and directly rewarded in non-monetary ways that closely mirror employee to employer relationships. Though we lack definitive data on it, this relationship and its repercussions deservers future study.
