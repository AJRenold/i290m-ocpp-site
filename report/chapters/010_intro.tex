\section{Introduction}

Open collaboration and peer production systems comprise a significant
part of the Internet's infrastructure and content.
Practicioners within these systems recognize the importance of
\emph{community}\footnote{Fogel, K., Producing Open Source Software, \url{http://producingoss.com/en/technical-infrastructure.html}} in developing and sustaining these resources.
Paradigmatically, an open source software project is created and maintained by an on-line community.
The community engages in productive dialog using communication tools and contributes to
a common pool \cite{ostrom1990}, in a process that is sometimes called ``private-collective innovation'' \cite{vonhippel2003oss}.
It has been suggested that the core features of such collaborative
communities extend to other communities of practice, such as those
surrounding wikis, open data, and citizen science, as well.

Our work is a contribution to the understanding of  open collaborative
communities and the underlying peer-production labor organization \cite{benkler2002}
We focus on the joining process and on how
it integrates in the broader issue of community management.
The authors of this report are all students and instructors in a
course at UC Berkeley's School of Information, entitled ``Open Collaboration and Peer Production'' (i290m).
All students have spent the semester participating in an open
collaborative community of their choosing and reporting back
their observations.
We have together developed a survey about our experiences joining
these projects, and the projects' organization and demographics.
By administering this survey to ourselves, we have surfaced data
that samples across a wide range of communities and researcher experiences.

\mysubsubsection{Discovering a Group to Join}
One of the central objectives of this report is to investigate factors that bring projects and potential contributors together. Prior to successfully joining and contributing to a project, newcomers must become aware of projects that are active and open to (or actively seeking) help, with available opportunities that resonate with the potential contributor's motivation to be part of an open source project. In the survey discussed here, we posed and analyzed questions relating to project outreach, contributor incentives, and the quality of the joining experience in order to gain novel perspective on the relationships between these factors for newcomers to a range of project types.

This report has also been \emph{written} collaboratively.
The 27 respondents \todo{my guess is there are rather 25 students and respondents} to our survey are all also authors of this paper.
We have attempted to organize ourselves according
to the best practices of open collaborative communities as discussed in literature.


In Section 2 we elaborate on the context of how we have written this report, as an exploration of this novel method is one of our main research contributions.
Section 3 provides some background literature which we draw from in
our survey design and analysis.
Section 4 explains our research methods, including quantitative
reporting and survey design.
Section 5 presents the results of our empirical work.
We discuss implications of our work in Section 6, and conclude. 


