\section{Introduction}

Open collaboration and peer production systems comprise a significant
part of the Internet's infrastructure and content.
Practicioners within these systems recognize the importance of
\emph{community}\footnote{Fogel, K., Producing Open Source Software, \url{http://producingoss.com/en/technical-infrastructure.html}} in developing and sustaining these resources.
Paradigmatically, an open source software project is created and maintained by an on-line community.
The community engages in productive dialog using communication tools and contributes to
a common pool \cite{ostrom1990}, in a process that is sometimes called ``private-collective innovation'' \cite{vonhippel2003oss}.
It has been suggested that the core features of such collaborative
communities extend to other communities of practice, such as those
surrounding wikis, open data, and citizen science, as well \todo{by whom?}.

Our work is a contribution to the understanding of  open collaborative
communities and the underlying peer-production labor organization \cite{benkler2002} \todo{don't now what this reference tries to say}.
A critical aspect is the joining and management of these communities. There exists previous efforts in these domains, which have studied joining into these communities and impacts of the joining \cite{vonKrogh2003,Baldwin2006} \todo{more refs would be nice}.
We also focus on the joining process and on how
it integrates in the broader issue of community management.
We contribute by exploring this issue from a novice aspect and by examining non-coding communities, such as citizen science. The former aspect is important, as novel recent developers have different kind of problems than those experienced by the developers \cite{Begel2008}. The latter represent an emerging form of open collaboration, and worth of studies as such.

Our context for this study is a semester long course at UC Berkeley's School of Information, entitled ``Open Collaboration and Peer Production'' (i290m). The course itself applied many open source principles, for example
the students and instructors wrote this report in collaboration.
The students have spent the semester participating in an open
collaborative community of their choosing and reporting
their observations.
We have collaboratively developed a survey about our experiences joining
these projects, and the projects' organization and demographics. The total number of responders is 24, who are also taking part in the report writing.
By administering this survey to ourselves, we have collected data
that samples across a wide range of communities and researcher experiences.

\mysubsubsection{Discovering a Group to Join}
One of the central objectives of this report is to investigate factors that bring projects and potential contributors together. Prior to successfully joining and contributing to a project, newcomers must become aware of projects that are active and open to (or actively seeking) help, with available opportunities that resonate with the potential contributor's motivation to be part of an open source project. In the survey discussed here, we posed and analyzed questions relating to project outreach, contributor incentives, and the quality of the joining experience in order to gain novel perspective on the relationships between these factors for newcomers to a range of project types.\todo{I would just remove whole section}

%This report has also been \emph{written} collaboratively.
%The 24 respondents to our survey are all also authors of this paper.
%We have attempted to organize ourselves according
%to the best practices of open collaborative communities as discussed in literature.

In Section 2 we elaborate on the context of how we have written this report, as an exploration of this novel method is one of our main research contributions.
Section 3 provides some background literature which we draw from in
our survey design and analysis.
Section 4 explains our research methods, including quantitative
reporting and survey design.
Section 5 presents the results of our empirical work.
We discuss implications of our work in Section 6, and conclude. 


