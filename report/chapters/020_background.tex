\section{Background}
\label{background}
A decade ago, open source software has appeared as an original self-organized ``Bazaar" way of producing software by opposition to commonly known ``Cathedral" top-down management \cite{raymond1999}. Like many surprising new phenomena it has also attracted research in management \cite{vonkrogh2006pro}, economics \cite{tirole2002some}, entrepreneurial legal studies \cite{benkler2002},  quantitative sociology \cite{crowston2005social} and even in complex systems physics \cite{maillart2008,tessone2011}. Open source software development has been understood as a private-collective innovation model in which developers gain both from their own (private) contributions as well as the (collective) work done by others \cite{vonhippel2003oss}. Despite notorious arising successes (e.g. Linux, Apache, Mozilla) and early recognition that fast growing Internet threats could be tackled better with full-disclosure (i.e. open source code) approaches to find and fix security holes in software (by opposition to non-disclosure  proprietary source code) \cite{}, it remained unclear how open source could prevail on proprietary business models.  The (apparent) benevolent commitment of large communities of developers was the most striking questions \cite{}. It is probably still the most fundamental question \cite{benkler2011leviathan}. But since the open source model has developed far beyond software programming as open collaboration for natural language knowledge production (e.g. Wikipedia), for industrial designs (3d printing, cars) \cite{raasch2009,pearce2012}, we at least know that open collaboration works well and probably far beyond most initial expectations. 

As companies get attracted by  ``bazaar" open collaboration and \linebreak peer-production models for innovation \cite{} and even production \cite{hamel2011first}, research tends to shift interest towards creating and maintaining user communities \cite{vonHippel2001} in closer collaboration with business  \cite{bonaccorsi2004ais} or more recently even completely integrated solutions within company boundaries \cite{}. For instance, one important requirement for getting a job at Facebook and in an increasing amount of Silicon Valley companies is precisely to have some open source software development experience, not only because the quality  of the code produced by the candidate can be checked {\it ex-ante}, but also because it helps build a sense of community in the company. In this precise case not only the expertise but also the social norms are internalized ! indeed motivation and social practice in open source projects have been found to be strongly related \cite{robert2006,vonKrogh2012}.

Open collaboration projects are rarely a completely horizontal bazaar. Most projects, when they reach a critical mass of contributors eventually self-organize with more or less implicit norms and sometimes with very well documented governance rules. It appears the process of shaping institutions goes along with the project development \cite{O'Mahony2007}. And in turn, the governance structure plays a role in the community structure, in particular how people join, contribute and eventually step down the project \cite{}. It is critical for a project survival to onboard enough contributors to offset those who leave. Governance structures can have deep influence on this, positive or negative. For instance, Wikipedia suffers from onboarding problems due to an over rigid governance system, which cannot account for community renewal \cite{halfacker2013}.

The number of software and non-software open collaboration projects has literally exploded with countless projects on Wikis and other online collaboration platforms such as Github or Sourceforge. For various intrinsic and extrinsic motivations to participate , like hedonic pleasure, reputation seeking or even to build a publicly available portfolio of achievements \cite{hars2001} , open collaboration projects attract the attention of an growing number of potential contributors, making even pressing the issue of tensions between (i) the necessity for projects to engage new contributors to offset natural attrition of ``old" developers, (ii)  the governance of continuously  renewing and growing communities, and (iii) its consequences for the on-boarding process of newcomers.

Enriched by the authors' immersion in the open collaboration project joining process over one semester, this collective work reports on the challenges of joining and how open collaboration communities should better manage it. Yet it is not the main aim here to describe its underlying production process,  the present report is also a collective work, which has been an important challenge for all the authors.  

