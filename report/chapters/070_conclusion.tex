\section{Conclusion}

This report originates form UC Berkeley School of Information course on peer-production and open collaboration. The course covered both the theoretical literature on open collaboration and practice based experiences from contributions. This report combines these two approaches together and examines the course students experiences as contributors in open source or open collaboration projects.

Our experiences explored joining scripts and organization structures further. Both of these have been discussed in the academic literature \cite{x,y,z}, and our work confirms and deepens these analysis. The first part of engagement is the steps to join in a peer production community, or joining scripts. There was variation on our experiences on these, especially in the level of documentation and organization support given to joining. Our experiences highlight the need for human-to-human communication in these critical initial steps: personal contact is an efficient outreach method and also needed to help and guide participants into their first contributions. Naturally this also can be seen as a problem related to the quality of documentation: based on our experience one in four projects did not provide the necessarily documentation to support contributions.

Regarding the project governance, our sample highlighted that benevolent dictatorship is the most prominent model of management. However, the management was supported by task management: visibility and ownership of tasks was seen helpful indicator on the community and also helping in the joining process.

Last we analyzed business models and funding. Our sample included mix of projects funded by academic grants\footnote{Note that this is most likely due to the context of this course being in an active research university.}, donations or corporation supported. These funding options have impact on the choice of licensing: academic and and donations supported open source projects used more ''ideological" licenses, such as GNU GPL and BSD where as corporations used most prominently Apache license. However, based on our experiences, the license is not an important factor for choosing which projects to contribute on -- for us atleast.

Naturally this study is limited, but suggests some interesting directions for future study. Experiences of students trying to join both very new projects (only a year or two old) and more established projects seemed to yield different results.  We would like to see future research directed towards illuminating the differences in joining experiences between participants who take part in projects of different ages and stages. 

One participant in a very young project explained: 

\begin{quote}
[...] it feels much more manageable to eventually gain an overall grasp of both the overall structure and many details for this type of early-stage project than if I were to try to contribute to a large, long-running and already widely used project.
\end{quote}